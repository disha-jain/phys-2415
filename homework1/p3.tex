\documentclass[../homework.tex]{subfiles}

\pagestyle{fancy}
\fancyhf{}
\rhead{Hanzhi Zhou}
\lhead{Physics 2415 Homework}
\cfoot{\thepage}

\begin{document}
\subsection{Problem 1.3}
\subsubsection*{a)}
\setlength{\leftskip}{1cm}
\paragraph*{i)}
Upward if $z_0 > \frac{h}{2}$ since there're more charges at the bottom, downward if $z_0 < \frac{h}{2}$ since there're more charges at the top
\paragraph*{ii)}
When $z_0 = \frac{h}{2}$. This point is the center of the cylindrical shell, and the electrical forces should be zero by symmetry.
\paragraph*{iii)}
When $z_0 >>h$, the cylindrical shell is a ring with radius $R$. Therefore,
\begin{equation*}
    \hspace*{1cm} E = \frac{1}{4\pi \epsilon_0} \frac{Qz_0}{(z_0^2 + R^2)^\frac{3}{2}} = \frac{1}{4\pi \epsilon_0} \frac{\sigma 2 \pi R h z_0}{(z_0^2 + R^2)^\frac{3}{2}} = \frac{\sigma R h}{2 \epsilon_0} \frac{z_0}{(z_0^2 + R^2)^\frac{3}{2}}
\end{equation*}
\indent When $z_0 >> h$ and $z_0 >> R$, the cylindrical shell acts like a point charge.
\begin{equation*}
    \hspace*{1cm} E = \frac{1}{4\pi \epsilon_0} \frac{\sigma 2\pi R h}{z_0^2} = \frac{\sigma R h}{2 \epsilon_0} \frac{1}{z_0^2}
\end{equation*}
\setlength{\leftskip}{0cm}
\subsubsection*{b)}
\begin{equation*}
    dQ = \sigma 2\pi R dz
\end{equation*}
\subsubsection*{c)}
\begin{equation*}
    d = |{z_0 - z}|
\end{equation*}
\subsubsection*{d)}
\begin{equation*}
    dE = \frac{1}{4\pi \epsilon_0} \frac{d(dQ)}{(d^2 + R^2)^\frac{3}{2}} = \frac{1}{4\pi \epsilon_0} \frac{(z_0 - z)\sigma 2 \pi R}{((z_0-z)^2 + R^2)^\frac{3}{2}} dz = \frac{\sigma R}{2\epsilon_0} \frac{z_0 - z}{((z_0-z)^2 + R^2)^\frac{3}{2}} dz
\end{equation*}
\subsubsection*{e)}
\begin{equation*}
    E = \frac{\sigma R}{2\epsilon_0} \int_{0}^{h} \frac{z_0 - z}{((z_0-z)^2 + R^2)^\frac{3}{2}} dz
\end{equation*}
\subsubsection*{f)}
\indent \indent By calculator,
\begin{equation*}
    E = \frac{\sigma R}{2\epsilon_0} \left(
    \frac{1}{\sqrt{R^2 + (z_0 - h)^2}} - \frac{1}{\sqrt{R^2 + z_0^2}}
    \right)
\end{equation*}
\indent The expectations \textbf{i)} and \textbf{ii)} are all met.
\subsubsection*{g)}
\indent \indent Rewriting the formula in \textbf{f)}
\begin{align*}
    E & = \frac{\sigma R}{2\epsilon_0} \left[
        \frac{1}{z_0 - h} \left(1 + \frac{R^2}{(z_0 - h)^2}\right)^{-\frac{1}{2}} - \frac{1}{z_0} \left(1 + \frac{R^2}{z_0^2}\right)^{-\frac{1}{2}}
    \right]
\end{align*}
\indent By first order binomial approximation, $(1 + x)^n \approx 1 + nx$
\begin{align*}
    E & = \frac{\sigma R}{2\epsilon_0} \left[
        \frac{1}{z_0 - h} \left(1 - \frac{1}{2}\frac{R^2}{(z_0 - h)^2}\right) - \frac{1}{z_0} \left(1 - \frac{1}{2}\frac{R^2}{z_0^2}\right)
    \right] \\[10pt]
      &= \frac{\sigma R}{2\epsilon_0} \left[
        \frac{1}{z_0 - h} - \frac{1}{z_0} - \frac{1}{2}\frac{R^2}{(z_0 - h)^3} + \frac{1}{2} \frac{R^2}{z_0^3}
    \right]
\end{align*}
\indent Because $z_0$ is large, we ignore the terms containing the cube of $z_0$. 
\begin{align*}
    E & = \frac{\sigma R}{2\epsilon_0} \left[
        \frac{1}{z_0 - h} - \frac{1}{z_0}
    \right] \\[10pt]
      & = \frac{\sigma R}{2\epsilon_0} \frac{h}{(z_0 - h) z_0} \\[10pt]
      & \approx \frac{\sigma R}{2\epsilon_0} \frac{h}{z_0^2}
\end{align*}
% Using the substitution $y = x \tan{\theta}$, $dy = x \sec^2{\theta} d\theta$
% \begin{align*}
%     \int \frac{dy}{(x^2 + y^2)^\frac{3}{2}} &= 
%     \int \frac{x \sec^2{\theta}}{(x^2 + x^2 \tan^2{\theta})^\frac{3}{2}} d\theta \\
%     &= \int \frac{x \sec^2{\theta}}{(x^2 \sec^2{\theta})^\frac{3}{2}} d\theta \\
%     &= \frac{1}{x^2} \int \cos{\theta} d\theta \\
%     &= \frac{1}{x^2} \sin{\theta} + C
% \end{align*}
% Since $y = x \tan{\theta}$, $\theta = \arctan{\frac{y}{x}}$
% \begin{align*}
%     \frac{1}{x^2} \sin{\theta} = \frac{1}{x^2} \sin{\arctan{\frac{y}{x}}} = \frac{y}{x^2 \sqrt{x^2 + y^2}}
% \end{align*}
% The fact that $\sin{\arctan{\frac{y}{x}}} = \frac{y}{\sqrt{x^2 + y^2}} $ comes from simple geometry. You draw a right-angled triangle with adjacent and opposite equal to $x$ and $y$ and you'll see.


\end{document}