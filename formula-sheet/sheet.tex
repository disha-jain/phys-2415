\documentclass[12pt,landscape]{article}
\usepackage{multicol}
\usepackage{calc}
\usepackage{ifthen}
\usepackage[a4paper, landscape]{geometry}
\usepackage[fleqn]{amsmath}
\usepackage{amsthm,amsfonts,amssymb}
\usepackage{color,graphicx,overpic}
\usepackage{hyperref}

\pdfinfo{
  /Title (sheet.pdf)
  /Creator (TeX)
  /Producer (pdfTeX 1.40.0)
  /Author (Hanzhi Zhou)
  /Subject (Formula Sheet)
  /Keywords (pdflatex, latex, pdftex, tex)}

% This sets page margins to .5 inch if using letter paper, and to 1cm
% if using A4 paper. (This probably isn't strictly necessary.)
% If using another size paper, use default 1cm margins.
\ifthenelse{\lengthtest { \paperwidth = 11in}}
    { \geometry{top=.5in,left=.5in,right=.5in,bottom=.5in} }
    {\ifthenelse{ \lengthtest{ \paperwidth = 297mm}}
        {\geometry{top=1cm,left=1cm,right=1cm,bottom=1cm} }
        {\geometry{top=1cm,left=1cm,right=1cm,bottom=1cm} }
    }

% Turn off header and footer
\pagestyle{empty}

% Redefine section commands to use less space
\makeatletter
\renewcommand{\section}{\@startsection{section}{1}{0mm}%
                                {-1ex plus -.5ex minus -.2ex}%
                                {0.5ex plus .2ex}%x
                                {\normalfont\large\bfseries}}
\renewcommand{\subsection}{\@startsection{subsection}{2}{0mm}%
                                {-1explus -.5ex minus -.2ex}%
                                {0.5ex plus .2ex}%
                                {\normalfont\normalsize\bfseries}}
\renewcommand{\subsubsection}{\@startsection{subsubsection}{3}{0mm}%
                                {-1ex plus -.5ex minus -.2ex}%
                                {1ex plus .2ex}%
                                {\normalfont\small\bfseries}}
\makeatother

% Define BibTeX command
\def\BibTeX{{\rm B\kern-.05em{\sc i\kern-.025em b}\kern-.08em
    T\kern-.1667em\lower.7ex\hbox{E}\kern-.125emX}}

% Don't print section numbers
\setcounter{secnumdepth}{0}


\setlength{\parindent}{0pt}
\setlength{\parskip}{0pt plus 0.5ex}

%My Environments
\newtheorem{example}[section]{Example}
% -----------------------------------------------------------------------

\begin{document}
\raggedright
\footnotesize
\begin{multicols}{3}

% multicol parameters
% These lengths are set only within the two main columns
%\setlength{\columnseprule}{0.25pt}
\setlength{\premulticols}{1pt}
\setlength{\postmulticols}{1pt}
\setlength{\multicolsep}{1pt}
\setlength{\columnsep}{2pt}
\setlength{\mathindent}{0pt}
\newcommand{\overbar}[1]{\mkern 1.5mu\overline{\mkern-1.5mu#1\mkern-1.5mu}\mkern 1.5mu}

\begin{center}
     \Large{\underline{PHYS 2415 Formula Sheet}} \\
\end{center}

\section{E Field \& Potential}
\begin{equation*}
    \Delta V = V_B - V_A = \frac{\Delta U}{q} = -\int_{A}^{B} \vec{E} \cdot d\vec{l}
\end{equation*}
\begin{equation*}
    \Delta U = U_b - U_a = q(V_b - V_a)
\end{equation*}
\begin{equation*}
    V = k\frac{Q}{r} = k \int \frac{dq}{r}
\end{equation*}
\begin{equation*}
    \vec{E} = - \vec{\nabla} V 
\end{equation*}
\begin{equation*}
    \vec{F} = q\vec{E}
\end{equation*}
\begin{equation*}
    U_{\text{system}} = k \sum_{\text{pairs i j}} \frac{q_i q_j}{r_{ij}}
\end{equation*}
\begin{equation*}
    E_{\text{rod}} = k \frac{\lambda L}{x \sqrt{x^2 + L^2 / 4}} = \frac{\lambda}{2\pi \epsilon_0 x }
\end{equation*}
\begin{equation*}
    E_{\text{ring}} = k \frac{Qx}{(x^2+R^2)^\frac{3}{2}}
\end{equation*}
\begin{equation*}
    E_{\text{nonconducting disk plate}} = \frac{\sigma}{2\epsilon_0} \left[1 - \frac{z}{\sqrt{z^2 + R^2}}\right] \approx \frac{\sigma}{2\epsilon_0}
\end{equation*}
\begin{equation*}
    E_{\text{conducting plane}} = \frac{\sigma}{\epsilon_0}
\end{equation*}
\begin{equation*}
    \text{dipole moment } p = Ql
\end{equation*}
\begin{equation*}
    \tau = \vec{p} \times \vec{E} = p E \sin{\theta}
\end{equation*}
\begin{equation*}
    W = \int_{\theta_1}^{\theta_2} \tau d\theta = -pE \int_{\theta_1}^{\theta_2} \sin{\theta} d\theta = p E (\cos{\theta_2} - \cos{\theta_1})
\end{equation*}

\section{Capacitance \& DC Circuit}
\begin{equation*}
    C_{\text{pplate}} = K \epsilon_0 \frac{A}{d}
\end{equation*}
\begin{equation*}
    W = U = \frac{1}{C} \int_{0}^{Q} q dq = \frac{1}{2}\frac{Q^2}{C} = \frac{1}{2} CV^2 = \frac{1}{2}QV
\end{equation*}
\begin{equation*}
    u = \frac{1}{2} \epsilon_0 \vec{E}^2 (J / m^3)
\end{equation*}
\begin{equation*}
    R = \rho \frac{l}{A}
\end{equation*}
\begin{equation*}
    \rho = \rho_0 [1 + \alpha(T - T_0)]
\end{equation*}
\begin{equation*}
    V = V_0 \sin{2\pi f} t = V_0 \sin{\omega t}
\end{equation*}
\begin{equation*}
    I_0 \sin{\omega t}
\end{equation*}
\begin{equation*}
    \overbar{P} = \frac{1}{2} I_0^2 R = \frac{1}{2} \frac{V_0^2}{R}= I_{\text{rms}} V_{\text{rms}}
\end{equation*}
\begin{equation*}
    \overbar{I^2} = \frac{1}{2} I_0^2
\end{equation*}
\begin{equation*}
    \overbar{V}^2 = \frac{1}{2} V_0^2
\end{equation*}
\begin{equation*}
    I_{\text{rms}} = \sqrt{\overbar{I^2}} = \frac{I_0}{\sqrt{2}}
\end{equation*}
\begin{equation*}
    V_{\text{rms}} = \sqrt{\overbar{V^2}} = \frac{V_0}{\sqrt{2}}
\end{equation*}
\begin{equation*}
    j = \frac{I}{A} = -n e \vec{v}_d 
\end{equation*}
\begin{equation*}
    n = \frac{N}{V}
\end{equation*}
\begin{equation*}
    \varepsilon = IR + \frac{Q}{C}
\end{equation*}
\begin{equation*}
    Q = C\varepsilon \left(1 - e^{\frac{-t}{RC}} \right)
\end{equation*}
\begin{equation*}
    Q = Q_0 e^{\frac{-t}{RC}}
\end{equation*}

% \subsection{xCode}
% Subsetction text

% \section{Section 3}

% You can even have references
\rule{0.3\linewidth}{0.25pt}
\scriptsize
\bibliographystyle{abstract}
\bibliography{refFile}
\end{multicols}
\end{document}