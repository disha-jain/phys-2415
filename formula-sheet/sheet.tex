\documentclass[12pt,landscape]{article}
\usepackage{multicol}
\usepackage{calc}
\usepackage{ifthen}
\usepackage[a4paper, landscape]{geometry}
\usepackage[fleqn]{amsmath}
\usepackage{amsthm,amsfonts,amssymb}
\usepackage{color,graphicx,overpic}
\usepackage{hyperref}
\usepackage{bm}
\usepackage{esint}

\pdfinfo{
  /Title (sheet.pdf)
  /Creator (TeX)
  /Producer (pdfTeX 1.40.0)
  /Author (Hanzhi Zhou)
  /Subject (Formula Sheet)
  /Keywords (pdflatex, latex, pdftex, tex)}

% This sets page margins to .5 inch if using letter paper, and to 1cm
% if using A4 paper. (This probably isn't strictly necessary.)
% If using another size paper, use default 1cm margins.
\ifthenelse{\lengthtest { \paperwidth = 11in}}
    { \geometry{top=.75in,left=.75in,right=.75in,bottom=.75in} }
    {\ifthenelse{ \lengthtest{ \paperwidth = 297mm}}
        {\geometry{top=1.5cm,left=1.5cm,right=1.5cm,bottom=1.5cm} }
        {\geometry{top=1.5cm,left=1.5cm,right=1.5cm,bottom=1.5cm} }
    }

% Turn off header and footer
\pagestyle{empty}

% Redefine section commands to use less space
\makeatletter
\renewcommand{\section}{\@startsection{section}{1}{0mm}%
                                {-1ex plus -.5ex minus -.2ex}%
                                {0.5ex plus .2ex}%x
                                {\normalfont\large\bfseries}}
\renewcommand{\subsection}{\@startsection{subsection}{2}{0mm}%
                                {-1explus -.5ex minus -.2ex}%
                                {0.5ex plus .2ex}%
                                {\normalfont\normalsize\bfseries}}
\renewcommand{\subsubsection}{\@startsection{subsubsection}{3}{0mm}%
                                {-1ex plus -.5ex minus -.2ex}%
                                {1ex plus .2ex}%
                                {\normalfont\small\bfseries}}
\makeatother

% Define BibTeX command
\def\BibTeX{{\rm B\kern-.05em{\sc i\kern-.025em b}\kern-.08em
    T\kern-.1667em\lower.7ex\hbox{E}\kern-.125emX}}

% Don't print section numbers
\setcounter{secnumdepth}{0}


\setlength{\parindent}{0pt}
\setlength{\parskip}{0pt plus 0.5ex}

%My Environments
\newtheorem{example}[section]{Example}
% -----------------------------------------------------------------------

\begin{document}
\raggedright
\footnotesize
\begin{multicols}{4}

    % multicol parameters
    % These lengths are set only within the two main columns
    %\setlength{\columnseprule}{0.25pt}
    \setlength{\premulticols}{1pt}
    \setlength{\postmulticols}{1pt}
    \setlength{\multicolsep}{1pt}
    \setlength{\columnsep}{2pt}
    \setlength{\mathindent}{0pt}
    \newcommand{\overbar}[1]{\mkern 1.5mu\overline{\mkern-1.5mu#1\mkern-1.5mu}\mkern 1.5mu}

    \begin{center}
        \Large{\underline{PHYS 2415 Formula Sheet}} \\
        \large By Hanzhi Zhou
    \end{center}

    \section{Electricity}
    \begin{equation*}
        E = \frac{1}{4 \pi \epsilon} \frac{Q}{r^2}
    \end{equation*}
    \begin{equation*}
        \Delta V = V_B - V_A = \frac{\Delta U}{q} = -\int_{A}^{B} \vec{\mathbf{E}} \cdot d\vec{\mathbf{l}}
    \end{equation*}
    \begin{equation*}
        \Delta U = U_b - U_a = q(V_b - V_a)
    \end{equation*}
    \begin{equation*}
        V = k\frac{Q}{r} = k \int \frac{dq}{r}
    \end{equation*}
    \begin{equation*}
        \vec{\mathbf{E}} = - \vec{\nabla} V
    \end{equation*}
    \begin{equation*}
        \vec{\mathbf{F}} = q\vec{\mathbf{E}}
    \end{equation*}
    \begin{equation*}
        U_{\text{system}} = k \sum_{\text{pairs i j}} \frac{q_i q_j}{r_{ij}}
    \end{equation*}
    \begin{equation*}
        \oiint \vec{\mathbf{E}} \cdot d \vec{\mathbf{A}} = \frac{Q}{\epsilon_0} = \frac{1}{\epsilon_0} \iiint \rho dV
    \end{equation*}
    \begin{equation*}
        E_{\text{rod}} = k \frac{\lambda L}{x \sqrt{x^2 + L^2 / 4}} \approx \frac{\lambda}{2\pi \epsilon_0 x }
    \end{equation*}
    \begin{equation*}
        E_{\text{ring}} = k \frac{Qx}{(x^2+R^2)^\frac{3}{2}}
    \end{equation*}
    \begin{equation*}
        E_{\text{disk}} = \frac{\sigma}{2\epsilon_0} \left[1 - \frac{z}{\sqrt{z^2 + R^2}}\right] \approx \frac{\sigma}{2\epsilon_0}
    \end{equation*}
    \begin{equation*}
        E_{\text{conducting plane}} = \frac{\sigma}{\epsilon_0}
    \end{equation*}
    \begin{equation*}
        \text{dipole moment } p = Ql
    \end{equation*}
    \begin{equation*}
        \tau = \vec{\mathbf{p}} \times \vec{\mathbf{E}} = p E \sin{\theta}
    \end{equation*}
    \begin{align*}
        W & = \int_{\theta_1}^{\theta_2} \tau d\theta = -pE \int_{\theta_1}^{\theta_2} \sin{\theta} d\theta \\
          & = p E (\cos{\theta_2} - \cos{\theta_1}) = -\vec{\mathbf{p}} \cdot \vec{\mathbf{E}}
    \end{align*}
    \begin{equation*}
        C_{\text{pplate}} = K \epsilon_0 \frac{A}{d}
    \end{equation*}
    \begin{equation*}
        U = \frac{1}{C} \int_{0}^{Q} q dq = \frac{1}{2}\frac{Q^2}{C} = \frac{1}{2} CV^2 = \frac{1}{2}QV
    \end{equation*}
    \begin{equation*}
        u = \frac{1}{2} \epsilon_0 \vec{\mathbf{E}}^2 (J / m^3)
    \end{equation*}
    \begin{equation*}
        R = \rho \frac{l}{A}
    \end{equation*}
    \begin{equation*}
        \rho = \rho_0 [1 + \alpha(T - T_0)]
    \end{equation*}
    \begin{equation*}
        V = V_0 \sin{2\pi f} t = V_0 \sin{\omega t}
    \end{equation*}
    \begin{equation*}
        I_0 \sin{\omega t}
    \end{equation*}
    \begin{equation*}
        \overbar{P} = \frac{1}{2} I_0^2 R = \frac{1}{2} \frac{V_0^2}{R}= I_{\text{rms}} V_{\text{rms}}
    \end{equation*}
    % \begin{equation*}
    %     \overbar{I^2} = \frac{1}{2} I_0^2
    % \end{equation*}
    % \begin{equation*}
    %     \overbar{V}^2 = \frac{1}{2} V_0^2
    % \end{equation*}
    \begin{equation*}
        I_{\text{rms}} = \sqrt{\overbar{I^2}} = \frac{I_0}{\sqrt{2}}
    \end{equation*}
    \begin{equation*}
        V_{\text{rms}} = \sqrt{\overbar{V^2}} = \frac{V_0}{\sqrt{2}}
    \end{equation*}
    \begin{equation*}
        j = \frac{I}{A} = -n e \vec{\mathbf{v}}_d
    \end{equation*}
    Number of electrons over volume
    \begin{equation*}
        n = \frac{N}{V}
    \end{equation*}
    \begin{equation*}
        \varepsilon = IR + \frac{Q}{C}
    \end{equation*}
    \begin{equation*}
        Q = C\varepsilon \left(1 - e^{\frac{-t}{RC}} \right)
    \end{equation*}
    \begin{equation*}
        Q = Q_0 e^{\frac{-t}{RC}}
    \end{equation*}

    \subsection{Magnetism}
    \begin{equation*}
        \vec{\mathbf{F}} = q \vec{\mathbf{v}} \times \vec{\mathbf{B}}\ \     \vec{\mathbf{F}} = I \vec{\boldsymbol{l}} \times \vec{\mathbf{B}}
    \end{equation*}
    \begin{equation*}
        qvB = \frac{mv^2}{r} \Rightarrow r = \frac{mv}{qB}
    \end{equation*}
    \begin{equation*}
        \vec{\boldsymbol{\tau}} = NI \vec{\mathbf{A}} \times \vec{\mathbf{B}} = \vec{\boldsymbol{\mu}} \times \vec{\mathbf{B}}
    \end{equation*}
    \begin{equation*}
        U = \int \tau d\theta = -\mu B \cos \theta = -\vec{\boldsymbol{\mu}} \cdot \vec{\mathbf{B}}
    \end{equation*}
    \begin{equation*}
        \vec{\mathbf{B}} = \frac{\mu_0}{4\pi} \int \frac{I d \vec{\boldsymbol{l}} \times \hat{\boldsymbol{r}}}{r^2} = \frac{\mu_0 N I R^2}{2(R^2+x^2)^\frac{3}{2}} \text{ Coil}
    \end{equation*}
    \begin{equation*}
        B_{\text{Solenoid}} = \frac{1}{2} \frac{\mu_0 N I}{l}
    \end{equation*}
    \begin{equation*}
        B_{\text{long wire}} = \frac{\mu_0 I}{2 \pi r}\,\, (\text{from Ampere's law})
    \end{equation*}
    \begin{equation*}
       \mathcal{X}_\mu = \frac{\mu - \mu_0}{\mu_0} \text{ magnetic susceptibility}
    \end{equation*}
    \begin{equation*}
        \frac{V_S}{V_P} = \frac{N_S}{N_P}
    \end{equation*}
    \begin{equation*}
        \varepsilon = \oint \vec{\mathbf{E}} \cdot d \vec{\boldsymbol{l}} = - \frac{\partial}{\partial t} \oiint \vec{\mathbf{B}} \cdot d \vec{\mathbf{A}}
    \end{equation*}
    \begin{equation*}
        \varepsilon_1 = -M \frac{dI_2}{dt},\,\,\,\varepsilon_2 = -M \frac{dI_1}{dt}
    \end{equation*}
    \begin{equation*}
        L = \frac{N \Phi_B}{I},\,\,\, \varepsilon = -N \frac{d\Phi_B}{dt} = -L \frac{dI}{dt}
    \end{equation*}
    \begin{equation*}
        U = \frac{1}{2} L I^2 = \frac{1}{2} \frac{B^2}{\mu_0} A l,\,\, \mu = \frac{1}{2} \frac{B^2}{\mu_0}
    \end{equation*}
    \begin{equation*}
        I = I_0 e^{-\frac{t}{\tau}},\,\, \tau = \frac{L}{R}
    \end{equation*}
    \begin{equation*}
        Z_{RLC} = \sqrt{R^2 + \left(\omega L - \frac{1}{\omega C}\right)^2}
    \end{equation*}
    \begin{equation*}
        \cos \phi = \frac{V_{R0}}{V_0} = \frac{R}{Z}
    \end{equation*}
    \begin{equation*}
        \oint \vec{\mathbf{B}} \cdot d \vec{\boldsymbol{l}} = \mu_0 I_{encl} + \mu_0 \epsilon_0 \frac{\partial}{\partial t} \oiint \vec{\mathbf{E}} \cdot d\vec{\mathbf{A}}
    \end{equation*}
    \begin{equation*}
        \oiint \vec{\mathbf{B}} \cdot d \vec{\mathbf{A}} = 0
    \end{equation*}
    \section{Waves}
    \begin{equation*}
        v = \sqrt{\frac{F_T}{\mu}} = \sqrt{\frac{E}{\rho}}\,\,, \mu = \frac{m}{l}
    \end{equation*}
    \begin{equation*}
        E = 2\pi^2 m f^2 A^2 = 2\pi^2 \rho S v t f^2 A^2
    \end{equation*}
    \begin{equation*}
        E = 2 \pi^2 \mu v t f^2 A^2
    \end{equation*}
    \begin{equation*}
        \text{Intensity } I = \frac{\overbar{P}}{S}
    \end{equation*}
    \begin{equation*}
        \frac{\partial^2 D}{\partial x^2} = \frac{\mu}{F_T} \frac{\partial^2 D}{\partial t^2}
    \end{equation*}
    \begin{equation*}
        E = E_0 \sin (kx - \omega t)
    \end{equation*}
    \begin{equation*}
        k = \frac{2\pi}{\lambda}, \,\, \omega = 2\pi f,\,\, f \lambda = \frac{\omega}{k} = v
    \end{equation*}
    \begin{equation*}
        E_0 = c B_0
    \end{equation*}
    \begin{equation*}
        \mu = \frac{1}{2}\epsilon_0 E^2 + \frac{1}{2} \frac{B^2}{\mu_0} = \epsilon_0 E^2 = \frac{B^2}{\mu_0}
    \end{equation*}
    \begin{equation*}
        \vec{\mathbf{S}} = \frac{1}{\mu_0} (\vec{\mathbf{E}} \times \vec{\mathbf{B}}) = \frac{1}{2} \epsilon_0 c E_0^2 = \frac{1}{2} \frac{c}{\mu_0} B_0^2
    \end{equation*}
    \begin{equation*}
        \overbar{S} = \frac{E_{rms}B_{rms}}{\mu_0} = \frac{E_0 B_0}{2 \mu_0}
    \end{equation*}
    \begin{equation*}
        P_{\text{refl}} = \frac{2}{Ac} \frac{dU}{dt} = \frac{2 \overbar{S}}{c}
    \end{equation*}
    \begin{equation*}
        c = \frac{1}{\mu_0 \epsilon_0}
    \end{equation*}

    \newpage

    \section{Geometric Optics}
    \begin{equation*}
        \frac{1}{f} = \frac{1}{d_i} + \frac{1}{d_o}
    \end{equation*}
    \begin{equation*}
        f = \frac{R}{2}
    \end{equation*}
    \begin{equation*}
        m = -\frac{d_i}{d_o}
    \end{equation*}
    Sign convention
    \begin{enumerate}
        \item $d_o$ positive if object at side where the light comes from, negative otherwise
        \item $d_i$ positive if image at the opposite side where the light comes from, negative otherwise.
        \item $f$ positive for converging lens, negative for diverging lens, infinity for mirror
    \end{enumerate}
    Lensmaker's equation. Note: if concave, $R$ is negative.
    \begin{equation*}
        \frac{1}{f} = (n - 1) \left(\frac{1}{R_1} + \frac{1}{R_2}\right)
    \end{equation*}
    \begin{equation*}
        n_1 \sin\theta_1 = n_2 \sin \theta_2
    \end{equation*}
    \begin{equation*}
        \sin \theta_c = \frac{n_2}{n_1}
    \end{equation*}
    \begin{equation*}
        n = \frac{c}{\sqrt{\mu \epsilon}}
    \end{equation*}
    \begin{equation*}
        \lambda_n = \frac{v}{f} = \frac{c}{nf} = \frac{\lambda}{n}
    \end{equation*}
    where $\lambda_n$ is the wavelength in the material with index of refraction $n$.
    \begin{equation*}
        P = \frac{1}{f} \text{ Units: dipoter. $1 D = 1 m^{-1}$}
    \end{equation*}
    \section{Diffraction and Interference}
    \subsection{Single-slit}
    Position of \textbf{minima}. Note that $D$ is the width of the slit and $m=0$ corresponds to the central maximum.
    \begin{equation*}
        D \sin \theta = m \lambda,\,\, m = \pm 1, \pm 2
    \end{equation*}
    \begin{equation*}
        I_{\theta} = I_0 \left[
            \frac{\sin \left(
                \frac{\pi D \sin \theta}{\lambda}
            \right)}{\left(
                \frac{\pi D \sin \theta}{\lambda}
            \right)}
        \right]^2
    \end{equation*}
    \subsection{Double-slit}
    Constructive interference (bright)
    \begin{equation*}
        d \sin \theta = m \lambda,\,\, m = 0, \pm 1, \pm 2...
    \end{equation*}
    Destructive interference (dark)
    \begin{equation*}
        d \sin \theta = \left(m + \frac{1}{2} \right) \lambda,\,\, m = 0, \pm 1, \pm 2...
    \end{equation*}
    \begin{equation*}
        I_\theta = I_0 \cos^2 \left(\frac{\delta}{2}\right)
    \end{equation*}
    \begin{equation*}
        \delta = \frac{2\pi}{\lambda} d \sin \theta
    \end{equation*}
    \subsection{Thin-film}
    If light reflects from the surface where $n_1 < n_2$, there will be $180^\circ$ phase shift ($\frac{1}{2}\lambda$~wavelength).

    Constructive interference
    \begin{equation*}
        \frac{2tn}{\lambda_{\text{vacuum}}} = m + \frac{1}{2}
    \end{equation*}
    Destructive interference
    \begin{equation*}
        \frac{2tn}{\lambda_{\text{vacuum}}} = m
    \end{equation*}
    where $t$ is the thickness and $n$ is the index of refraction of the material
    \subsection{Diffraction grating}
    Positions of principle \textbf{maxima}. $d$ is the width between slits.
    \begin{equation*}
        d \sin \theta = m \lambda
    \end{equation*}
    \subsection{Other}
    Angular resolution (Rayleigh criterion). $D$: diameter.
    \begin{equation*}
        \theta = \frac{1.22 D}{\lambda}
    \end{equation*}
    \rule{0.3\linewidth}{0.25pt}
    \scriptsize
    \bibliographystyle{abstract}
    \bibliography{refFile}
\end{multicols}
\end{document}