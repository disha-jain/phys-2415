\documentclass[../homework.tex]{subfiles}

\pagestyle{fancy}
\fancyhf{}
\rhead{Hanzhi Zhou}
\lhead{Physics 2415 Homework}
\cfoot{\thepage}

\begin{document}
\subsection{Problem 5.1}
\subsubsection*{a)}
\begin{align*}
    d\vec{B} &= \frac{\mu_0}{4\pi} \frac{I d\vec{l} \times \hat{r}}{r^2} = \frac{\mu_0}{4\pi}\frac{I d\vec{l} \times \vec{r}}{r^3} \\
    &= \frac{\mu_0}{4\pi}\frac{I}{\sqrt{R^2 + y^2}^3}
    \begin{bmatrix}
        0 \\
        -dy  \\
        0
    \end{bmatrix} \times
    \begin{bmatrix}
        -R \\
        -y  \\
        0
    \end{bmatrix} \\
    &= \frac{\mu_0}{4\pi}\frac{IR}{(R^2 + y^2)^\frac{3}{2}} dy~\hat{k}
\end{align*}
\begin{equation*}
    \vec{B} = \frac{\mu_0}{4\pi}\int_{0}^{\infty} \frac{IR}{(R^2 + y^2)^\frac{3}{2}} dy~\hat{k} = \frac{I\mu_0}{4\pi R}\hat{k}
\end{equation*}

\subsubsection*{b)}
\begin{align*}
    d\vec{B} &= \frac{\mu_0}{4\pi}\frac{I}{\sqrt{R^2}^3}
    \begin{bmatrix}
        \cos(\theta + \frac{\pi}{2}) \\
        \sin(\theta + \frac{\pi}{2})  \\
        0
    \end{bmatrix} d\theta \times R
    \begin{bmatrix}
        -\cos \theta \\
        -\sin \theta  \\
        0
    \end{bmatrix} \\
    &= \frac{\mu_0}{4\pi}\frac{I}{R} d\theta~\hat{k}
\end{align*}
\begin{equation*}
    \vec{B} = \frac{\mu_0}{4\pi}\int_{\pi}^{\frac{3}{2}\pi} \frac{I}{R} d\theta~\hat{k} = \frac{\mu_0I}{8\pi R}~\hat{k}
\end{equation*}

\subsubsection*{c)}
\begin{align*}
    d\vec{B} &= \frac{\mu_0}{4\pi}\frac{I}{\sqrt{R^2 + x^2}^3}
    \begin{bmatrix}
        dx \\
        0  \\
        0
    \end{bmatrix} \times
    \begin{bmatrix}
        -x \\
        R  \\
        0
    \end{bmatrix} \\
    &= \frac{\mu_0}{4\pi}\frac{IR}{(R^2 + x^2)^\frac{3}{2}} dx~\hat{k}
\end{align*}
\begin{equation*}
    \vec{B} = \frac{\mu_0}{4\pi}\int_{0}^{\infty} \frac{IR}{(R^2 + x^2)^\frac{3}{2}} dx~\hat{k} = \frac{I\mu_0}{4\pi R}\hat{k}
\end{equation*}

\subsubsection*{d)}
\begin{equation*}
    \vec{B}_{\text{total}} = 2\frac{I\mu_0}{4\pi R}\hat{k} + \frac{\mu_0I}{8\pi R}~\hat{k} = \frac{9\mu_0I}{16\pi R}~\hat{k} = 9 \frac{4\pi \times 10^{-7} \times 0.5}{16 \pi \times 2 \times 10^{-2}} = \frac{9}{16} \times 10^{-5}  T \hat{k}
\end{equation*}

\subsubsection*{e)}\indent\indent
They are already indicated by unit vectors
\end{document}